\documentclass{mshw}
\begin{document}
\acadclass{18.06 - Fall 2013}
\title{Problem Set 1}
\author{Miles Steele}

\problem{1}{\#19 from 2.1}
\begin{gather*}
\overbrace{
  \pmat{
    1 & 0 & 0 \\
    0 & 1 & 0 \\
    1 & 0 & 1 \\
  }
}^E
\pmat{ x \\ y \\ z } = \pmat{ x \\ y \\ z + x }
\\
\overbrace{
  \pmat{
    1 & 0 & 0 \\
    0 & 1 & 0 \\
    -1 & 0 & 1 \\
  }
}^{E^{-1}}
\pmat{ x \\ y \\ z } = \pmat{ x \\ y \\ z - x }
\\
E \pmat{ 3 \\ 4 \\ 5 } = \pmat{ 3 \\ 4 \\ 8 }
\\
E^{-1} E \pmat{ 3 \\ 4 \\ 5 }
= E^{-1} \pmat{ 3 \\ 4 \\ 8 }
= \pmat{ 3 \\ 4 \\ 5 }
\end{gather*}

\problem{2}{\#19 from 2.2}
\begin{gather*}
\pmat{
  1 & 4 & -2 \\
  1 & 7 & -6 \\
  0 & 3 & q \\
}
\pmat{ x \\ y \\ z }
= \pmat{ 1 \\ 6 \\ t }
\\
\text{row}_2 \mathrel{{-}{=}} \text{row}_1
\\
\pmat{
  1 & 4 & -2 \\
  0 & 3 & -4 \\
  0 & 3 & q \\
}
\pmat{ x \\ y \\ z }
= \pmat{ 1 \\ 5 \\ t }
\\
\text{row}_3 \mathrel{{-}{=}} \text{row}_2
\\
\pmat{
  1 & 4 & -2 \\
  0 & 3 & -4 \\
  0 & 0 & q + 4 \\
}
\pmat{ x \\ y \\ z }
= \pmat{ 1 \\ 5 \\ t-5 }
\end{gather*}

\subproblem{Singular System ($q$)}
If $q = -4$ then the third equation is $0 = t-5$, which exposes no information about $z$.\\
Thus if $q = -4$ then the system is singular.

\subproblem{No Solutions ($t$)}
\emph{What value of $t$ results in the system having no solutions?}

If $q = -4$ and $t = 5$ then the third equation is $0 = 0$. Thus the value of $z$ is irrelevant and the system has infinitely many solutions.

\subproblem{Solution where $z=1$}
Find the solution which has $z = 1$.
\begin{gather*}
q = -3 \\ t = 6
\\
\pmat{
  1 & 4 & -2 \\
  0 & 3 & -4 \\
  0 & 0 & 1 \\
}
\pmat{ x \\ y \\ z }
= \pmat{ 1 \\ 5 \\ 1 }
\\
z = 1 \\
3 y - 4 z = 5 \\
y = 3 \\
x + 4 y - 2 z = 1 \\
x = -9 \\
\pmat{ x \\ y \\ z } = \pmat{ -9 \\ 3 \\ 1 }
\end{gather*}


\problem{3}{Graph Intersection}
The graph of
\[ y = a + bx + c\sin\f{\pi x}{2} \]
passes through the points $(0, 2), (1, 6), (2, 12)$.

Each pair of points represents an equation.
\begin{gather*}
\pmat{
  1 & 0 & 0 \\
  1 & 1 & 1 \\
  1 & 2 & 0 \\
}
\pmat{ a \\ b \\ c }
= \pmat{ 2 \\ 6 \\ 12 }
\end{gather*}
swap row$_2$ with row$_3$
\begin{gather*}
\pmat{
  1 & 0 & 0 \\
  1 & 2 & 0 \\
  1 & 1 & 1 \\
}
\pmat{ a \\ b \\ c }
= \pmat{ 2 \\ 12 \\ 6 }
\end{gather*}
That looks enough like a Gaussian Eliminated matrix.
Back substitution, here we come.
\begin{gather*}
a = 2 \\
2 + 2 b = 12 \\
b = 5 \\
2 + 5 + c = 6 \\
c = -1
\\
\pmat{ a \\ b \\ c } = \pmat{ 2 \\ 5 \\ -1 }
\end{gather*}

\problem{4}{\#20 from 2.4}
\begin{multicols}{2}
\begin{align*}
A &= \pmat{
  0 & 2 & 0 & 0 \\
  0 & 0 & 2 & 0 \\
  0 & 0 & 0 & 2 \\
  0 & 0 & 0 & 0
} \\
A^2 &= \pmat{
  0 & 0 & 4 & 0 \\
  0 & 0 & 0 & 4 \\
  0 & 0 & 0 & 0 \\
  0 & 0 & 0 & 0
} \\
A^3 &= \pmat{
  0 & 0 & 0 & 8 \\
  0 & 0 & 0 & 0 \\
  0 & 0 & 0 & 0 \\
  0 & 0 & 0 & 0
} \\
A^4 &= \pmat{
  0 & 0 & 0 & 0 \\
  0 & 0 & 0 & 0 \\
  0 & 0 & 0 & 0 \\
  0 & 0 & 0 & 0
} \\
v &= \pmat{ x \\ y \\ z \\ t }\\
\end{align*}
\begin{align*}
A v &= \pmat{
  0 & 2 & 0 & 0 \\
  0 & 0 & 2 & 0 \\
  0 & 0 & 0 & 2 \\
  0 & 0 & 0 & 0
}
\pmat{ x \\ y \\ z \\ t }
= \pmat{ 2 y \\ 2 z \\ 2 t \\ 0 }
\\
A^2 v &= \pmat{
  0 & 0 & 4 & 0 \\
  0 & 0 & 0 & 4 \\
  0 & 0 & 0 & 0 \\
  0 & 0 & 0 & 0
}
\pmat{ x \\ y \\ z \\ t }
= \pmat{ 4 z \\ 4 y \\ 0 \\ 0 }
\\
A^3 v &= \pmat{
  0 & 0 & 0 & 8 \\
  0 & 0 & 0 & 0 \\
  0 & 0 & 0 & 0 \\
  0 & 0 & 0 & 0
}
\pmat{ x \\ y \\ z \\ t }
= \pmat{ 8 t \\ 0 \\ 0 \\ 0 }
\\
A^4 v &= \pmat{
  0 & 0 & 0 & 0 \\
  0 & 0 & 0 & 0 \\
  0 & 0 & 0 & 0 \\
  0 & 0 & 0 & 0
}
\pmat{ x \\ y \\ z \\ t }
= \pmat{ 0 \\ 0 \\ 0 \\ 0 }
\\
\end{align*}
\end{multicols}


\problem{5}{\#23 from 2.4}
\subproblem{}
\emph{Find a nonzero matrix $A$ for which $A^2 = 0$}.
\begin{align*}
A &= \pmat{
 0 & 1 \\
 0 & 0 \\
} \\
A^2 &= \pmat{
  0 & 0 \\
  0 & 0 \\
}
\end{align*}

\subproblem{}
\emph{Find a matrix that has $A^2 \neq 0$ but $A^3 = 0$.}
\begin{align*}
A &= \pmat{
  0 & 2 & 0 & 0 \\
  0 & 0 & 2 & 0 \\
  0 & 0 & 0 & 0 \\
  0 & 0 & 0 & 0
} \\
A^2 &= \pmat{
  0 & 0 & 4 & 0 \\
  0 & 0 & 0 & 0 \\
  0 & 0 & 0 & 0 \\
  0 & 0 & 0 & 0
} \\
A^3 &= \pmat{
  0 & 0 & 0 & 0 \\
  0 & 0 & 0 & 0 \\
  0 & 0 & 0 & 0 \\
  0 & 0 & 0 & 0
}
\end{align*}


\newpage
\problem{6}{\#3 from 2.5}
\begin{gather*}
\pmat{ 10 & 20 \\ 20 & 50 } \pmat{ x \\ y } = \pmat{ 1 \\ 0 }
\\
\text{row}_2 \mathrel{{-}{=}} 2 \text{row}_1
\\
\pmat{ 10 & 20 \\ 0 & 10 } \pmat{ x \\ y } = \pmat{ 1 \\ -2 }
\\
y = -0.2 \\
10 x + 20 (-0.2) = 1 \\
x = 0.5 \\
\pmat{ x \\ y } = \pmat{ 0.5 \\ -0.2 0}
\end{gather*}

\begin{gather*}
\pmat{ 10 & 20 \\ 20 & 50 } \pmat{ t \\ z } = \pmat{ 0 \\ 1 }
\\
\text{row}_2 \mathrel{{-}{=}} 2 \text{row}_1
\\
\pmat{ 10 & 20 \\ 0 & 10 } \pmat{ t \\ z } = \pmat{ 0 \\ 1 }
\\
z = 0.1 \\
10 t + 20 (0.1) = 0 \\
t = -0.2 \\
\pmat{ t \\ z } = \pmat{ -0.2 \\ 0.1 }
\end{gather*}

\begin{gather*}
A^{-1} = \pmat{ x & t \\ y & z } = \pmat{ 0.5 & -0.2 \\ -0.2 & 0.1 }
\end{gather*}


\newpage
\problem{7}{\#3 from 2.6}
% TODO

\newpage
\problem{8}{Upper Triangular Squares}
\begin{lstlisting}
julia> {triu(ones(n,n))^2 for n=1:5}
5-element Array{Any,1}:
 1x1 Array{Float64,2}:
 1.0
 2x2 Array{Float64,2}:
 1.0  2.0
 0.0  1.0
 3x3 Array{Float64,2}:
 1.0  2.0  3.0
 0.0  1.0  2.0
 0.0  0.0  1.0
 4x4 Array{Float64,2}:
 1.0  2.0  3.0  4.0
 0.0  1.0  2.0  3.0
 0.0  0.0  1.0  2.0
 0.0  0.0  0.0  1.0
 5x5 Array{Float64,2}:
 1.0  2.0  3.0  4.0  5.0
 0.0  1.0  2.0  3.0  4.0
 0.0  0.0  1.0  2.0  3.0
 0.0  0.0  0.0  1.0  2.0
 0.0  0.0  0.0  0.0  1.0
\end{lstlisting}

It appears that the bottom triangle remains zeros but the upper triangle increases towards to the top-right where it reaches $n+1$.
When trying to understand why this is, I found it helpful to think about splitting up the rows. The first row of $A^2$ is $A[1;:]*A$. $A^2[2;:] = A[2;:]*A$, etc.


\newpage
\problem{9}{Order Exploration}
Cool.
\begin{lstlisting}
julia> [mean([order(rperm(j)) for i=1:10000]) for j=1:15]
15-element Array{Float64,1}:
  1.0
  1.4963
  2.1783
  2.7911
  3.9452
  4.5654
  6.2151
  7.3896
  9.043
 10.8293
 13.6837
 15.7355
 19.0345
 22.0164
 24.9219
\end{lstlisting}
I found that the average order depends greatly upon the size of the matrix. This makes some intuitive sense as their is more space to for the $P$ to wander around before finding its way back to $I$.


\newpage
\problem{10}{Speedy Computers}
\begin{gather*}
\underbrace{f(N)}_\text{fp.computations} = \f23 {\underbrace{N}_\text{vars}}^3 \\
N_\text{blue} = 12,681,215 \\
\underbrace{S_\text{blue}}_\text{speed of IBM} = 1 \\
S_\text{ALAM} = 2 \\
\f{f(N)}S = T \\
\f{f(N_\text{blue})}{S_\text{blue}} = T
= \f{f(N_\text{ALAM})}{S_\text{ALAM}} \\
f(N_\text{blue}) = \f{f(N_\text{ALAM})}2 \\
\f23 {N_\text{blue}}^3 = \f{{N_\text{ALAM}}^3}3 \\
{\p{2 {N_\text{blue}}^3}}^{\f13} = {N_\text{ALAM}} \\
N_\text{ALAM} \approx 15,977,329 \text{ variables.}
\end{gather*}

\end{document}
